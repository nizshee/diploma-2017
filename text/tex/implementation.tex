\section{Реализация}

В этом разделе наше внимание сместится на осбенности реализации процессов
связанных с типами в Scala Plugin.
Особенное внимание будет уделено спецификации scala \cite{scala_spec}.
Мы проследим как соответсвующие понятия от туда переносятся в Scala Plugin.

До этого момента, когда мы говорили о работе плагина, то использовали
нейтральное слово процесс.
Теперь нужно вспомнить, что изначальной задачей было явно визуализировать
работу связанную с типами, которую плагин делает неявно.
Перечислим интересующие нас процессы.

Базовым вопросом является сравнение двух типов.
В спецификации для этого вводятся три понятия: эквивалентность типов, сводимость
типов и слабая сводимость типов.
Эквивалентность означает что один тип мы в любом контексте можем заменить другим
типом, и это отношение наиболее понятно интуитивно.
Сводимость типов намного более интересна и используется во всех других процессах.
Ее мы рассмотрим в разделе~\ref{sec:conformance}.

Вывод типов  - что-то было

Разрешение перегрузок - ...

implicits - уже существует инструмент...
Динамические типы...

При этом не стоит забывать что Scala Plugin ровно как и компилятор scala
должны реализовывать спецификацию scala \cite{scala_spec}.
Смысл в последующей визуализации
Хочется сделать С уважением к спецификации
Однако не везде это возможно сделать
Где можно сделаем
Везде будет указано расхождение

Достаточно абстрактоно и упрощенно архитектуре плагина будет
рассмотрена в разделе \ref{sec:arch}.
Там же дается общая ифнормация о сводимости типов, выводе типов и поиске
наиболее специфичной перегрузки.

Более подробно о типах scala и их представлении в плагине.
В разделе \ref{sec:conformance} будут рассмотрены типы описанные спецификацией
scala на примере проверки их сводимости. Также внимание будет уделено
отличию от типов в scala plugin.

В разделе \ref{sec:infer} основной темой будет отличия вывода в плагине от
спецификации.

Завершит все  разрешение перегрузок \ref{sec:overloading}.

\subsection{Архитектура Scala Plugin}
\label{sec:arch}

Постоянные проверки типа Any, Nothing...
JavaArray
Object

Хиндли-Милнер \cite{hindley–milner}


\subsection{Проверка сводимости типов}
\label{sec:conformance}
Value Types

Non-Value Types

\subsubsection{Singleton Types}
\subsubsection{Type Projection}
\subsubsection{Type Designators}
\subsubsection{Parameterized Types}
\subsubsection{Compound Types}
\subsubsection{Existential Types}
\subsubsection{Method Types}
\subsubsection{Polymorphic Method Types}
Типы высших кайндов - куда-то
\subsubsection{Type Constructors}
Запустить, посмотреть во что выльется.

\subsection{Вывод типов}
\label{sec:infer}

Вывод типов локальный.
В спекцификации вывод типов так-то.
В плагине абсолютно иначе.
\subsection{Разрешение перегрузок функций}
\label{sec:overloading}
