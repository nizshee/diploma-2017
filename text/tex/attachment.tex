\vfill
\clearpage
\appendix

\stepcounter{section}
\hfill ПРИЛОЖЕНИЕ А
\begin{center}
  \textbf{Пример проверки сводимости}
\end{center}
\markboth{\MakeUppercase{}}{}
\addcontentsline{toc}{section}{Приложение А. Пример проверки сводимости}

\begin{figure}[h!]
  \centering
  \begin{subfigure}{\linewidth}
    \centering
    \includegraphics[width=.8\linewidth]{img/conformance1}
    % \caption{Неправильное сведение}
  \end{subfigure}

  \begin{subfigure}{\linewidth}
    \centering
    \includegraphics[width=.8\linewidth]{img/conformance2}
    % \caption{Правильное сведение}
  \end{subfigure}
  \caption{Проверка сводимости}
  \label{fig:conformance}
\end{figure}


\vfill
\clearpage
\newpage


\vfill
\clearpage
\appendix

\stepcounter{section}
\hfill ПРИЛОЖЕНИЕ Б
\begin{center}
  \textbf{Пример вывода типов}
\end{center}
\markboth{\MakeUppercase{}}{}
\addcontentsline{toc}{section}{Приложение Б. Пример вывода типов}

\begin{figure}[h!]
  \begin{minipage}{.5\linewidth}
  \centering
  \includegraphics[scale=.35]{img/infer1}
  \end{minipage}%
  \begin{minipage}{.5\linewidth}
  \centering
  \includegraphics[scale=.35]{img/infer3}
  \end{minipage}\par\medskip
  \centering
  \includegraphics[scale=.7]{img/infer2}

  % \centering
  % \begin{subfigure}{\linewidth}
  %   \centering
  %   \includegraphics[width=.7\linewidth]{img/infer1}
  %   \caption{Неправильное сведение}
  % \end{subfigure}
  %
  % \begin{subfigure}{\linewidth}
  %   \centering
  %   \includegraphics[width=.7\linewidth]{img/infer2}
  %   \caption{Правильное сведение}
  % \end{subfigure}
  %
  % \begin{subfigure}{\linewidth}
  %   \centering
  %   \includegraphics[width=.7\linewidth]{img/infer3}
  %   \caption{Правильное сведение}
  % \end{subfigure}
  \caption{Проверка сводимости}
  \label{fig:infer}
\end{figure}


\vfill
\clearpage
\newpage


\vfill
\clearpage
\appendix

\stepcounter{section}
\hfill ПРИЛОЖЕНИЕ В
\begin{center}
  \textbf{Пример выбора перегрузки функции}
\end{center}
\markboth{\MakeUppercase{}}{}
\addcontentsline{toc}{section}{Приложение В. Пример выбора перегрузки функции}

\begin{figure}[h!]
  \centering
  \begin{subfigure}{\linewidth}
    \centering
    \includegraphics[width=.8\linewidth]{img/conformance1}
    \caption{Неправильное сведение}
  \end{subfigure}

  \begin{subfigure}{\linewidth}
    \centering
    \includegraphics[width=.8\linewidth]{img/conformance2}
    \caption{Правильное сведение}
  \end{subfigure}
  \caption{Проверка сводимости}
  \label{fig:overloading}
\end{figure}
