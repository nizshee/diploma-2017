\section{Обзор}

Scala является популярным языком программирования.

Система типов.

Скала использует многоие штуки из теории типов.
\begin{itemize}
  \item Унифицированная система типов
  \item Параметризованные типы
  Вариантность.
  \item Структурные типы
  \item Типы высших кайндов
  \item Implicit
  \item Экзестенциальные типы
  \item Path-dependent types
\end{itemize}

Однако использование широких возможностей системы типы влечет к избыточости.
Все эти типы могут занимать много места.

Поэтому компилятор может дописывать типы за пользователя.
Компилятор может выводить типы за пользователя, что особенно удобно с
использованием лямбда-функций.
Т.е. типы не указываются явно, а компилятор делает неявное преобразование.

If you have, you might have wondered why the OCaml compiler does not produce
better error messages. The reason is: it can't.

Хотелось бы исеть штуку, которая поможет с этим.
Причем лучше чтобы она была доступна в какой-то интегрированной среде разработке.
Заметим что ИСР должны выполнять ту же работу по выводу типов что и компилятор.
Семантические ошибки требуют знания типов.
Существуют две подобные ИСР: Scala Plugin для intellij idea и плагин для
eclipse scala IDE.
Нужно определиться и выбрать одну из них.
Далее мы будем говорить про intellij idea.

Рассмотрим, какие бывают помощники, описывающие работу компилятора.

\subsection{Отладчик типов для языка программирования OCaml}
\label{sec:ocaml}

OCaml, как и большинство функциональных языков, использует мощную статическкую
типизацию.
Как и в scala компилятор берет на себя работу по выводу типов.
During type inference, the OCaml compiler finds a type conflict between
two subexpressions: one requires one type and the other provides a different
type. However, there is no way for the OCaml compiler to know which of the two
types is the right type for the programmer. Thus, the OCaml compiler arbitrarily
chooses one of them and reports it as a type error.

Сущетвует интересный проект, призванный помочь разработчикам.

The Type Debugger in this page improves on this situation by asking
questions to the programmer. From the answers to the questions, it understands
the programmer's intention and leads the programmer to the real source of the
type error. This idea of asking questions to obtain programmer's intention is
due to Shapiro's algorithmic program debugging and is used by Chitil to design
a type debugger for a small subset of Haskell.

We extend this approach to
support full OCaml by reusing the type inferencer in the OCaml compiler.

Существование подобного проекта составляет прецедент для подобных помощников.

\subsection{Scala type debugger}
\label{sec:typeDebugger}

Возникает логичный вопрос - существует ли подобный инструмент для языка
программирования
Есть такая штука.

Пара статей со ссылками, одна написана в соавторстве с Мартином Одерски.

Обладает рядом проблем - нужно тащить в репозиторий.
Но на момент написания работы не обновлялось в течение 5 лет.
Укзано что eclipse, но нет.

Последней трудностью будет то, что он все равно использует scala compiler.
Scala Plugin же строит занимается анализом кода самостоятельно,
без помощи компилятора.

\subsection{Show implicit parameters action}
\label{sec:showImplicit}

В качестве заключении рассмотрим как в рамках интегрированной среды
разработки помогать пользователю анализировать/моделировать работу выполняемую
компилятором.

В качестве примера мы будем использовать Show implicit parameters action.
Механизм implicit параметров.
Существуют определенные правила поиска соответвствующих параметров.

Эта штука показывает что да как.

\subsection{Постановка цели}

Можно привести три случая непостредственной пользы подобной штуки.

Понизить уровень вхождения.

Работа будет основана на работе плагина, а не компилятора.
Несмотря на, в некоторых нетривиальных случаях это может посочь.

Для людей которые должны работать с внутренним устройством плагина это может
помочь без необходимости непосредственной отладки.

\textbf{Цель:}
Сделать отладчик для процессов связанных с типами в рамках Scala Plugin.

\textbf{Задачи:}
\begin{itemize}
  \item Инструментировать Scala Plugin для сбора данных о рпоцемме работы с типами.
  \item
  \item
\end{itemize}

Стоит сделать замечание, в разделе~\ref{sec:instrumentation} будет более подробно
рассказано почему в качестве способа сбора данных было выбрано именно
инструментирование исходного кода плагина.
